\section{Future Work} \label{sec:future_work}

Two of the algorithm we have implemented has only been applied to one of the
two datasets we have worked with. In particular the random forest approach has
only been applied to the PAN 2015 data and the \gls{SVM} approach has only been
applied to the PAN 2013 data. The main reason for that, is that the random
forest approach requires a single known text per author and the \gls{SVM}
approach require multiple known texts per author. It would be interesting to
apply the algorithms to the opposite datasets and look at their performance
there. We can transform the dataset with only a single known text to a dataset
with multiple by splitting the known text into a collection of known texts.
Similarly we can transform the dataset with multiple known texts to a dataset
with a single known text by concatenating the known texts. By making those two
transformations we could have run all methods on all data and found results for
all of them.

We would also like to apply our different implemented methods to larger
datasets. We have found in this assignment that having more text available per
author improves the performance of our methods. All our implemented methods
performed better on the PAN 2013 dataset than on the PAN 2015 dataset. For
example there is a Danish company, MaCom, that produce software for turning in
and managing school assignments. MaCom has a large database containing several
texts for each student and they have an interest in authorship verification.
They specifically want to verify that assignments uploaded by students match
their previous assignments and are not written by someone else. MaCom's main
requirement is a method that has a high \gls{TPR}. The reason is that they
don't want to falsely accuse anyone of not having written their own assignment,
while it doesn't matter as much if they miss some assignment that is written by
someone else. When the \gls{TPR} is high it means that there is few \gls{FN}s
and many \gls{TP}s. \gls{FN}s are as described earlier when we say a text is
written by a different author while it is actually written by the same.

Since MaCom has much data available per student there case most closely match
that of the PAN 2013 dataset. The method we implemented with the best \gls{TPR}
on the PAN 2013 dataset was our \gls{SVM} approach. So it would be interesting
to apply that approach to MaCom's dataset.
