\section{Method}

\subsection{Feature Extraction}
Feature extraction from a text is the process of finding a vector that
represents the text. There are a lot of different features that can be
extracted and we use several of them. Specifically we use some character
level features, some word level features and som \gls{POS} tagging
features.

\begin{description}

    \item[Character n-grams:] We extract character n-grams (TODO: Fill in sizes)
    by extracting all substrings of length n from a text and computing the
    frequencies of the different substrings. We only use the K(TODO: specify)
    most frequent n-grams. The most frequent n-grams are found by concatenating
    all training text and computing frequencies in that.

    \item[Special character n-grams:] We extract special character n-grams of
    size (TODO: specify) by filtering all alphanumeric and space characters out
    of a string. We are then left with only special characters and we extract
    n-grams as described above from that string.

    \item[Word frequencies:] We extract the frequencies of the n(TODO: specify)
    most common words. We find the most common words from a concatenation of all
    the training data files and extract the frequencies of those words from the
    files one by one.

    \item[Word n-grams:] We extract word n-grams of size (TODO: specify). The
    n-grams are extracted by converting the text to a list of words. To split
    the text we use the \textit{nltk} python module.

    \item[POS tagging n-grams:] We use POS tag n-grams of size (TODO:specify).
    The POS tags are generated using Polyglot (TODO: citation) POS tagging
    engine. We then extract n-grams as described above.

\end{description}

% FROM maitra2015
%
%for example-
%Letter frequencies, N-gram frequencies, Function word usage, Vocabulary richness,
%Lexical richness, Distribution of syllables per word, Word frequencies, Hapax
%legomena, Hapax dislegomena, Word length distribution, Word collocations, Sentence
%length, Preferred word positions, Prepositional phrase structure, Distribution
%parts of speech, Phrasal composition grammar etc. [1][2][3][4][8][9]. The fact is that
%there is no such consensus on which stylometric features are applied to achieve the
%best results for authorship identification.
%
% Total Punctuation Count: This feature counts the number of total punctuation
%symbols used in a text, normalized by the word count in that text.
%
% Specific Punctuation Ratio: This is the ratio of the total number of specific
%punctuation symbols like comma (,), semicolon (;), question-mark (?), exclamation-mark
%(!), stop (.), slash (/), dash (-), colon (:) etc. to the total punctuation
%count.
%
% Long-sentence/ Short-sentence Ratio: Ratio of the long (length>12) or
%short (length<6) sentences to the total number of sentences is represented by
%this feature.
%
% Vocabulary Strength: We tried to capture the vocabulary strength of an author
%by calculating the ratio of the unique words used to the total number of
%words used in a text snippet.
%
% xPOS Frequency: In this feature, we try to capture the tendency of an author
%to use one or two particular types of POS that appear more frequently than
%the others, if there is any. So, we calculate the frequencies of each POS tag
%from texts and compare the known and unknown texts based on that.
%
% Starting POS Frequency: We try to list the POS tags that the author uses in
%the beginning of sentences according to their frequency and then compare
%them among the known and unknown documents to find a lexical pattern.
%For example, a particular author might have the tendency to start sentences
%with auxiliary verbs (example) or prepositions (in, for) unknowingly for a
%considerable number of sentences in the corpus. The feature also indicates
%the writing style of the author.
%In the above example, both known
%

% FROM pacheco2015
%
% Number of sentences: minimum, average and maximum number of sentences per
%paragraph and document
%
%Spacing: minimum, average and maximum number of consecutive spaces, number
% of consecutive spaces in the beginning/end of the line and number of empty lines
%
% Punctuation: minimum, average and maximum number of punctuation characters
%(.,;?¿!¡"’) per document, paragraph and sentence.
%
%  Lexical density: measure of how “dense” is the content, i.e, the ratio between each
%lexical category (nouns, adjectives, verbs and adverbs) divided by the total number
%of words
%
% Hapax: number of words that appear only one time and are only used by the current
%author.




% From Castro:2015
%1. Character
%(a) Tri-grams of characters (F1)
%(b) Quad-grams of characters (F2)
%(c) Word prefixes of size 2 (F3)
%(d) Word suffixes of size 2 (F4)
%2. Words
%(a) Uni-grams of words (F5)
%(b) Tri-grams of words (F6)
%3. Lemma and Part of Speech
%(a) Uni-grams of lemmas (F7)
%(b) Uni-grams of Part of Speech (F8)
%(c) Tri-grams of lemmas (F9)
%(d) Tri-grams of Part of Speech (F10)


% FROM bartoli2015b
% Sentence lengths (SL) We transform the document to a sequence of tokens, a token
%being a sequence of characters separated by one or more blank spaces. Next, we
%transform the sequence of tokens to a sequence of sentences, a sentence being a
%sequence of tokens separated by any of the following characters: .,;,:,!,?. We
%count the number of sentences whose length in tokens is n, with n ∈ {1, . . . , 40}:
%we obtain a feature for each value of n.)
%
% Sentence length ngrams (SG) We transform each document to a sequence of labels,
%where each label represents a full sentence and is chosen based on the sentence
%length (as described in the following). Next, we compute the ngram)
%
%Word richness (WR) We transform the document to a sequence of words as for the
%WG features group. Then we compute the ratio between the number of distinct
%words and the number of total w)
%
%Text shape ngrams (TG) We transform the document as follows: sequences of digits
%are replaced by the single character n; sequences of alphabetic characters are replaced
%by a single character: l if all the characters in the sequence are lowercase,
%u if only the first character is uppercase, w if at least two characters are uppercase;
%sequences of blank spaces are replaced by a single blank space; other characters are
%left unchanged.)

\subsection{Delta Method}
We have chosen to use the Delta method as a baseline method for our other
implementations. The Delta method is described by \cite{evert2015towards} and
consist of extracting word frequencies, applying a linear transformation to
those frequencies and using KNN with different distance metrics. There are a
number of parameters to choose. First of all the number of different words
to find frequencies for has to be chosen (it was originally chosen at 150
words \cite{evert2015towards}). Then the linear transformation can be chosen,
the usual transformation is a normalization to zero mean unit variance. And
finally the distance metric can be chosen. In our implementation we work with
n(TODO: specify) different word frequencies, the linear transformation is
a normalization of the data and the used distance metric is the Manhatten
distance.

TODO: Write result of our test.

% TODO: Describe other approaches we implement.

\subsection{Experiment}
