\section{Introduction} \label{sec:introduction}
Authorship verification is the process of verifying the authorship of a text.
You are given a set of texts known to be written by the author and an unknown
text that has to be classified as either written by the author or someone else.
The normal approach to the problem is to use stylometry to extract features from
the text and then some form of machine learning or statistical method to analyse
the data \cite{stamatos2009}. Many different text features have been proposed
to describe an author's writing style. That includes, but is not limited to,
character frequencies, word frequencies, vocabulary size, sentence length,
punctuation usage, character n-grams, word n-grams and \gls{POS}-tagging n-grams
\cite{stamatos2009}.

Authorship verification has been used for plagiarism control in danish secondary
schools \cite{hansen2014} and also has uses in civil law, criminal law and
computer forensics \cite{stamatos2009}.

In this article we explore state-of-the-art methods for authorship verification.
We extract several types of features including word frequencies, word n-grams
and \gls{POS}-tagging n-grams. We implement several algorithms to work on the
features. We start by implementing a baseline method which is a distance based
approach in which an unknown text is considered to be written by the closest
author. We then use the baseline method to compare to our other results. The
other results are obtained from several different methods. A Random Forest based
method, an extension of the baseline method and a \gls{SVM} based method. We
use data from two sources \cite{pan:2015} and \cite{pan:2013} which are two
instances of a yearly competition in digital text forensics. Since we use the
data from those two competitions we also compare our results to the results
obtained by others in the competition.
