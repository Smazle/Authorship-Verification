\section{Introduction}
Authorship verification is the process of verifying the authorship of a text.
You are given a set of texts known to be written by the author and an unknown
text that has to be classified as either written by the author or someone else.
The normal approach to the problem is to use stylometry to extract features from
the text and then some form of machine learning or statistical method to analyse
the data \cite{stamatos2009}. A lot of different text features has been proposed
as describing an authors writing style. That includes but is not limited to
character frequencies, word frequencies, vocabulary size, sentence length,
punctuation usage, character n grams, word n grams and \gls{POS} tagging n grams
\cite{stamatos2009}.

Authorship verification has been used for plagiarism control in danish secondary
schools \cite{hansen2014} and also has uses in civil law, criminal law and
computer forensics \cite{stamatos2009}.
