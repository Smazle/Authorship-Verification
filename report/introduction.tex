\section{Introduction}
Authorship verification is the process of verifying the authorship of a text.
You are given a set of texts known to be written by the author and an unknown
text that has to be classified as either written by the author or someone else.
The normal approach to the problem is to use stylometry to extract features from
the text and then some form of machine learning or statistical method to analyse
the data \cite{stamatos2009}. A lot of different text features has been proposed
as describing an authors writing style. That includes but is not limited to
character frequencies, word frequencies, vocabulary size, sentence length,
punctuation usage, character n-grams, word n-grams and \gls{POS} tagging n-grams
\cite{stamatos2009}.

Authorship verification has been used for plagiarism control in danish secondary
schools \cite{hansen2014} and also has uses in civil law, criminal law and
computer forensics \cite{stamatos2009}.

In this article we explore state of the art methods for authorship verification.
We extract several different types of features including word frequencies, word
n-grams and pos-tagging n-grams. We implement several different algorithms to
work on the features. We start by implementing a baseline method which is a
distance based approach where an unknown text is considered to be written by
the closest author. We then use the baseline method to compare to our other
results. We use data from two sources \cite{pan:2015} and \cite{pan:2013} which
are two instances of a yearly competition in digital text forensics. Since we
use PAN data we also compare our results to the results obtained by others in
the competition.

% TODO: Continue.
%The PAN2013 and PAN2015 datasets are very different. The PAN2013 dataset
%contains many texts from a few authors while the PAN2015 texts contains a single
%text from many authors.
