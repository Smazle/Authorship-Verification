\section{Feature Extraction}
%for example-
%Letter frequencies, N-gram frequencies, Function word usage, Vocabulary richness,
%Lexical richness, Distribution of syllables per word, Word frequencies, Hapax
%legomena, Hapax dislegomena, Word length distribution, Word collocations, Sentence
%length, Preferred word positions, Prepositional phrase structure, Distribution
%parts of speech, Phrasal composition grammar etc. [1][2][3][4][8][9]. The fact is that
%there is no such consensus on which stylometric features are applied to achieve the
%best results for authorship identification.
%
% Total Punctuation Count: This feature counts the number of total punctuation
%symbols used in a text, normalized by the word count in that text.
%
% Specific Punctuation Ratio: This is the ratio of the total number of specific
%punctuation symbols like comma (,), semicolon (;), question-mark (?), exclamation-mark
%(!), stop (.), slash (/), dash (-), colon (:) etc. to the total punctuation
%count.
%
% Long-sentence/ Short-sentence Ratio: Ratio of the long (length>12) or
%short (length<6) sentences to the total number of sentences is represented by
%this feature.
%
% Vocabulary Strength: We tried to capture the vocabulary strength of an author
%by calculating the ratio of the unique words used to the total number of
%words used in a text snippet.
%
% xPOS Frequency: In this feature, we try to capture the tendency of an author
%to use one or two particular types of POS that appear more frequently than
%the others, if there is any. So, we calculate the frequencies of each POS tag
%from texts and compare the known and unknown texts based on that.
%
% Starting POS Frequency: We try to list the POS tags that the author uses in
%the beginning of sentences according to their frequency and then compare
%them among the known and unknown documents to find a lexical pattern.
%For example, a particular author might have the tendency to start sentences
%with auxiliary verbs (example) or prepositions (in, for) unknowingly for a
%considerable number of sentences in the corpus. The feature also indicates
%the writing style of the author.
%In the above example, both known
%
% FROM maitra2015
