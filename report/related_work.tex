% TODO: Maybe call this previous work instead.
\section{Related Work}
% Describe stylometry.

% aalykke2016, hansen2014, stamatos2009.

PAN \footnote{\url{http://pan.webis.de/}} keeps a collection of shared tasks in
digital text forensics. In 2013, 2014 and 2015 the tasks focused on authorship
verification. In the 2015 task a dataset of authors with a set of texts were
given. Each author had a known text and an unknown text and the task was to
determine which of the unknown texts belonged to the same author as the known
text.

% Magnus.
% gomezadorno2015

% August.
% juanpablo2015

\cite{maitra2015} implemented a solution for the PAN 2015 task. They used a
collection of different features extracted from the text. The features were
based on punctuation, sentence length, vocabulary, character n grams and
Parts-of-Speech (POS) tagging. They trained a random forest classifier on the
features extracted and used that to determine whether or not the unknown texts
were written by the author. Their results were not overwhelming and they
commented that deep learning might make their results better.

\cite{pacheco2015} also proposed using a random forest for the PAN 2015 task.
They implemented two baseline models and one real model. The baseline models
were a simple distance metric with a trained cutoff point and a Gaussian Mixture
Model - Universal Background Model (UBM). The second baseline model is about
defining a general feature vector for all authors and a feature vector for
each specific author. To determine if a text from an unknown author belongs
to a given author you compute the distance between the texts feature set and
the universal and author specific feature set. If the unknown text is closer
to universal than to the author specific it is presumed to not be written by
the author. The used features in this case were number of stop words, number
of sentences, number of paragraphs, spacing, punctuation, word frequencies,
character frequencies, punctuation frequencies, lexical density, word diversity,
unique words and unique words over all authors. The main model were a random
forest and a UBM. A feature vector representing all documents and a feature
vector for each document were constructed. The vectors were then combined and
the result were fed to a random forest model. Their results were promising.

% Magnus.
% pacheco2015

% August.
% castro2015

% August.
% gutierrez2015

% Magnus.
% bartoli2015b
