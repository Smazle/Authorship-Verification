% TODO: Maybe call this previous work instead.
\section{Related Work}
% Describe stylometry.

% stamatos2009.

PAN \footnote{\url{http: // pan.webis.de /}} keeps a collection of shared tasks 
in
digital text forensics. In 2013, 2014 and 2015 the tasks focused on authorship
verification. In the 2015 task a dataset of authors with a set of texts were
given. Each author had a known text and an unknown text and the task was to
determine which of the unknown texts belonged to the same author as the known
text.

% Magnus.
% gomezadorno2015

% August.
% juanpablo2015
\cite{juanpablo2015} chose to perform their analysis on the syntactic level in
their submission for the PAN 2015 task. This was done using syntactic n - grams,
which was extracted used a syntactic analyser designed for the designated
language. This syntactic served to subvert the topic of text, as the focus
would be on the syntactic class of the n - grams extracted, rather than the word
it self.
After extracting the n - grams wanted, some light filtering was performed,
removing the less frequent n - grams. At this point they chose to represent 
their
n - grams frequencies, as a vector in space, allowing them to use the jaccard
distance to measure the difference between this newly introduced unknown text,
and its' proposed authors' known texts. When the similarity fell under a
certain threshold, the author was deemed non - valid.
This yielded somewhat mediocre results, and they themselves noted that a new
heuristic handling ill - constructed sentences would probably improved their
results, as the were just discarded in their case, which caused very low
accuracy in some cases. Additionally, more features describing other linguistic
layers, such as lexical, and syntactic feature would probably improve results
as well.

\cite{maitra2015} implemented a solution for the PAN 2015 task. They used a
collection of different features extracted from the text. The features were
based on punctuation, sentence length, vocabulary, character n grams and
\gls{POS} tagging. They trained a random forest classifier on the features
extracted and used that to determine whether or not the unknown texts were
written by the author. Their results were not overwhelming and they commented
that deep learning might make their results better.

\cite{pacheco2015} also proposed using a random forest for the PAN 2015 task.
They implemented two baseline models and one real model. The baseline models
were a simple distance metric with a trained cutoff point and a Gaussian Mixture
Model - Universal Background Model(UBM). The second baseline model is about
defining a general feature vector for all authors and a feature vector for
each specific author. To determine if a text from an unknown author belongs
to a given author you compute the distance between the texts feature set and
the universal and author specific feature set. If the unknown text is closer
to universal than to the author specific it is presumed to not be written by
the author. The used features in this case were number of stop words, number
of sentences, number of paragraphs, spacing, punctuation, word frequencies,
character frequencies, punctuation frequencies, lexical density, word diversity,
unique words and unique words over all authors. The main model were a random
forest and a UBM. A feature vector representing all documents and a feature
vector for each document were constructed. The vectors were then combined and
the result were fed to a random forest model. Their results were promising.

\cite{bartoli2015b} proposed yet another random forest based approach.
They didn't use a random forest classifier as \cite{maitra2015} and
\cite{pacheco2015} but a random forest regressor. The used features were word n
grams, character n grams, \gls{POS} tag n grams, word lengths, sentence lengths,
sentence length n grams, word richness \footnote{Word richness is number of
                                                 distinct words in a text 
divided by the total number of words.}, punctuation
n grams and text shape n grams. They then performed a feature selection and
normalization. They performed the final regression with both trees, a random
forest and an SVM. They ended up choosing the random forest as it performed the
best. Their results were very good being the best performer on Spanish texts.

% August.
% castro2015
\cite{castro2015} presents an approach that focuses more on the feature, than 
the algorithm applied to it. Using a set of 10 features spanning across 3 
different linguistic layer, the character layer, the lexical layer and the 
semantic layer. For each of these 10 features a vote is cast.
This vote is determined by, comparing the average similarity of the authors 
texts. In the scenario where a new unknown document is introduced, and wished 
validate the following happend.
For each author, the similarity of their texts are computed using one of the 10 
feature vectors as input for a chosen similarity function. The similarity for 
each of these author are then averaged to form the \texttt{Average Group 
Similarity}(AGS). The new document is then added to the group of documents of 
the proposed author, and the similarity of that group of documents is computed. 
If that similarity is above AGS, then a vote for this being the correct author 
is thrown.
However, before the vote for the feature is finalized, this is done with 3 
different similarity functions, Cosine, Dice and 1 - MinMax. The majority vote 
of those 3, determine what should be votes on that feature. This is done for 
each of the 10 feature, where in case of a tie vote(5 against 5) no decision is 
taken. This yielded some very good results, however some questions was raised 
as to the accuracy of documents on other genres.


% August.
% gutierrez2015

\cite{hansen2014} and \cite{aalykke2016} both describe usage of authorship
attribution methods in identifying authors of texts written in Danish secondary
school. \cite{aalykke2016} mainly used a distance based approach, he extracted
features and then used different distance metrics to compute the closest and
therefore best author. \cite{hansen2014} used SVM's for the author
classification they obtained an accuracy of 84\% .
